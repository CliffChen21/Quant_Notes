\documentclass{article}[a4]
\usepackage{amsmath,amsthm,amssymb}
\usepackage{xcolor}
% Set up the images/graphics package
\usepackage{graphicx}
\setkeys{Gin}{width=\linewidth,totalheight=\textheight,keepaspectratio}
\graphicspath{{figures/}}

\title{Historical Correlation Estimation}
\author{Cliff}

% Make prettier tables.
\usepackage{booktabs}

% The units package provides nice, non-stacked fractions and better spacing
% for units.
\usepackage{amsmath}

% The fancyvrb package lets us customize the formatting of verbatim
% environments.  We use a slightly smaller font.
\usepackage{fancyvrb}


% Small sections of multiple columns
\usepackage{multicol}



%%% Custom Commands
%---------------------------------------------------------------------------
% Concise referencing
\newcommand{\eqnref}[1]{\eqref{#1}}
\newcommand{\secref}[1]{Section \ref{#1}}
\newcommand{\figref}[1]{Figure \ref{#1}}
\newcommand{\lemref}[1]{Lemma \ref{#1}}
\newcommand{\corref}[1]{Corollary \ref{#1}}
\newcommand{\thmref}[1]{Theorem \ref{#1}}
% Real numbers
\newcommand{\Real}[1]{\mathbb{R}^{#1}}
% Complex numbers
\newcommand{\Complex}[1]{\mathbb{C}^{#1}}
% Integers
\newcommand{\Integer}[1]{\mathbb{Z}^{#1}}
% Rank operator
\DeclareMathOperator{\rank}{\textnormal{rank}}
% Vec operator
\newcommand{\vecop}{\textnormal{vec}}
% Norm
\newcommand{\norm}[1]{\left|\left|#1\right|\right|}
% Trace
\newcommand{\trace}{\textnormal{tr}}
% Range
\newcommand{\range}{\textnormal{range}}
% Partial derivative
\newcommand{\pd}[2]{\dfrac{\partial #1}{\partial #2}}
% Complete derivative
\newcommand{\dd}[2]{\dfrac{d #1}{d #2}}
% Complete derivative, second order
\newcommand{\dds}[2]{\dfrac{d^2 #1}{d {#2}^2}}
% Limit to N / N
\newcommand{\limover}[1]{\lim_{#1 \rightarrow \infty} \dfrac{1}{#1}}
% Display style sum
\newcommand{\dsum}{\displaystyle\sum}
% arg min and arg max
\newcommand{\argmax}[1]{\underset{#1}{\operatorname{arg~max}}}
\newcommand{\argmin}[1]{\underset{#1}{\operatorname{arg~min}}}
%---------------------------------------------------------------------------
\newtheorem{theorem}{Theorem}
\newtheorem{definition}{Definition}
\newtheorem{lemma}{Lemma}

\begin{document}

\maketitle
\begin{abstract}
	The document discussed the difference between weekly esimation and daily estimation on historical correlation with assumption that assets were traded continously.
\end{abstract}
\section{Assumptions}
For simplicity, we assumed the interest rates satisfy normal distribution and their covariance matrix were constant. Let $\mathbf{X},\mathbf{Y}$ be two normal random variables, and $X, Y$ be their sample path from 1 to T, which they can also be the sample paths of interest rates.\par 

\begin{equation}
\mathbf{X}\sim \mathcal{N}(\mu, \sigma^{2}),\quad \mathbf{Y}\sim \mathcal{N}(\mu ', \sigma'^{2})
\end{equation}\par 
And assmued there were N weeks in the history, then $T = 7N$, and\par 
\begin{equation}
\begin{aligned}
X &= (x_{1}, x_{2},x_{3},...,x_{7N})\\
Y &=(y_{1},y_{2},y_{3},...,y_{7N})
\end{aligned}
\end{equation}\par 
Since the interest rates in weekend are not visible, the daily raturn series can be expressed as below,\par 
\begin{equation}\label{daily}
\begin{aligned}
X^{w} &= (x_{2},x_{3},x_{4},x_{5},x_{6}+x_{7}+x_{8}, x_{9},x_{10}, x_{11},x_{12}, ...,x_{7N-2})\\
Y^{w} &= (y_{2},y_{3},y_{4},y_{5},y_{6}+y_{7}+y_{8}, y_{9},y_{10}, y_{11},y_{12}, ...,y_{7N-2})
\end{aligned}
\end{equation}\par 
And the weekly return was \par 
\begin{equation}\label{weekly}
\begin{aligned}
X^{w} &= (\sum_{m=0}^{6}x_{1+m}, \sum_{m=0}^{6}x_{2+m},\sum_{m=0}^{6}x_{3+m},...,\sum_{m=0}^{6}x_{7N-6+m})\\
Y^{w} &= (\sum_{m=0}^{6}y_{1+m}, \sum_{m=0}^{6}y_{2+m},\sum_{m=0}^{6}y_{3+m},...,\sum_{m=0}^{6}y_{7N-6+m})\\
\end{aligned}
\end{equation}\par 
\section{Sample Covariance}
Since we know, \par 
$$
Correl(X,Y) = \dfrac{Cov(X,Y)}{\sqrt{Var(X)}\sqrt{Var(Y)}}
$$\par 
and\par 
$$
Var(X) = Cov(X,X)
$$\par 
we could directly compared the covariance between these methods, and the sample covariance was \par 
\begin{equation}
Cov(X,Y) = \dfrac{\sum^{N}_{i=1}(X_{i}-\bar{X})(Y_{i}-\bar{Y})}{N-1}=\dfrac{\sum^{N}_{i=1}X_{i}Y_{i}}{N-1} - \dfrac{N\bar{X}\bar{Y}}{N-1}
\end{equation}
\section{Analysis}
\subsection{Daily Estimation}
Recall the formula (\ref{daily}), it can be generalized as \par 
\begin{equation}
\begin{aligned}
X^{d}&=({x_{n+1}, x_{n+2}, x_{n+3},x_{n+4},\sum_{m=0}^{2}x_{n+m+5}})_{n=1}^{N}\\
Y^{d}&=({y_{n+1}, y_{n+2}, y_{n+3},y_{n+4},\sum_{m=0}^{2}y_{n+m+5}})_{n=1}^{N}
\end{aligned}
\end{equation}\par 
\begin{equation}
\begin{aligned}
Cov(X^{d},Y^{d})&=\mathbb{E}\{X^{d}Y^{d}\} - \mathbb{E}\{X^{d}\}\mathbb{E}\{Y^{d}\} \\
\end{aligned}
\end{equation}
and its covariance is\par 
\begin{equation}
\begin{aligned}
\hat{Cov}(X^{d},Y^{d}) &= \dfrac{\sum_{n=1}^{N}\left(\sum_{i=1}^{4}x_{n+i}y_{n+i} + (\sum_{m=0}^{2}x_{n+m+5})(\sum_{m=0}^{2}y_{n+m+5})\right)}{5N-1} \\ &-\dfrac{5N}{5N-1}\left(\dfrac{\sum_{j=1}^{7N}x_{j}}{5N}\right)\left(\dfrac{\sum_{k=1}^{7N}y_{k}}{5N}\right)\\
&= \dfrac{\sum_{n=1}^{N}\left(\sum_{i=1}^{4}x_{n+i}y_{n+i} + \sum_{m=0}^{2}\sum_{p=0}^{2}x_{n+m+5}y_{n+p+5}\right)}{5N-1} \\ &-\dfrac{5N}{5N-1}\left(\dfrac{\sum_{j=1}^{7N}x_{j}}{5N}\right)\left(\dfrac{\sum_{k=1}^{7N}y_{k}}{5N}\right)
\end{aligned}
\end{equation}
Take an expectation,\par 
\begin{equation}
\begin{aligned}
\mathbb{E}\{\hat{Cov}(X^{d},Y^{d})\} &=\mathbb{E}\{\dfrac{\sum_{n=1}^{N}\left(\sum_{i=1}^{4}x_{n+i}y_{n+i} + \sum_{m=0}^{2}\sum_{p=0}^{2}x_{n+m+5}y_{n+p+5}\right)}{5N-1}\}\\
&=\dfrac{\mathbb{E}\{\sum_{n=1}^{N}\left(\sum_{i=1}^{4}x_{n+i}y_{n+i} + \sum_{m=0}^{2}x_{n+m+5}y_{n+m+5} \right)\}}{5N-1}\\
&+\dfrac{\mathbb{E}\{\sum_{m=0}^{2}\sum_{p=0,p\neq m}^{2}x_{n+m+5}y_{n+p+5}\}}{5N-1}\\
& = \dfrac{7N}{5N-1}Cov(x,y)
\end{aligned}
\end{equation}
Since \par 
\begin{equation}
\hat{Cov}(X,Y) = \dfrac{\sum_{i=1}^{7N}\mathbb{E}(x_{i}y_{i})}{7N-1} - \dfrac{7N}{7N-1}\left(\dfrac{\sum_{i=1}^{7N}\mathbb{E}(x_{i})}{7N}\right)\left(\dfrac{\sum_{i=1}^{7N}\mathbb{E}(y_{i})}{7N}\right)
\end{equation}

\subsection{Weekly Estimation}
Recall the formula (\ref{weekly}),\par 
\begin{equation}
\begin{aligned}
X^{w}&=(\sum_{m=0}^{6}x_{n+m})_{n=1}^{N}\\
Y^{w}&=(\sum_{m=0}^{6}y_{n+m})_{n=1}^{N}
\end{aligned}
\end{equation}\par 
and its covariance is \par 
\begin{equation}
\begin{aligned}
&\hat{Cov}(X^{w},Y^{w}) = \dfrac{\sum_{7N-6}^{n=1}\left(\sum_{m=0}^{6}x_{m+n}\sum_{m=0}^{6}y_{m+n}\right)}{7N-7} \\
&-\dfrac{7N-6}{7N-7}\left(\dfrac{\sum_{n=1}^{7N-6}\sum_{i=n}^{n+6}x_{i}}{7N-6}\right)\left(\dfrac{\sum_{n=1}^{7N-6}\sum_{i=n}^{n+6}y_{i}}{7N-6}\right)
\end{aligned}
\end{equation}
Take and expectation\par 
\begin{equation}
\begin{aligned}
\mathbb{E}\{\hat{Cov}(X^{w},Y^{w}) \} &= \mathbb{E}\{\dfrac{\sum_{7N-6}^{n=1}\left(\sum_{m=0}^{6}x_{m+n}\sum_{m=0}^{6}y_{m+n}\right)}{7N-7}\} \\
&- \mathbb{E}\left(\dfrac{7N-6}{7N-7}\left(\dfrac{\sum_{n=1}^{7N-6}\sum_{i=n}^{n+6}x_{i}}{7N-6}\right)\left(\dfrac{\sum_{n=1}^{7N-6}\sum_{i=n}^{n+6}y_{i}}{7N-6}\right)\right)\\
&=\mathbb{E}\{\dfrac{\sum_{7N-6}^{n=1}\left(\sum_{m=0}^{6}\sum_{p=0}^{6}x_{m+n}y_{p+n}\right)}{7N-7}\}
-\dfrac{36(7N-6)}{7N-7}\mathbb{E}\left(x\right)\mathbb{E}\left(y\right)\\
&=\mathbb{E}\{\dfrac{\sum_{7N-6}^{n=1}\left(\sum_{m=0}^{6}x_{m+n}y_{m+n} + \sum_{m=0}^{6}\sum_{p=0,p\neq m}^{6}x_{m+n}y_{p+n}\right)}{7N-7}\}\\
&-\dfrac{36(7N-6)}{7N-7}\mathbb{E}\left(x\right)\mathbb{E}\left(y\right)\\
& = \dfrac{\sum_{n=1}^{7N-6}\sum_{m=0}^{6}\mathbb{E}(xy) + 0}{7N-7}-\dfrac{6(7N-6)^{2}}{7N-7}\mathbb{E}\left(x\right)\mathbb{E}\left(y\right)\\
& = \dfrac{\sum_{n=1}^{7N-6}\sum_{m=0}^{6}(Cov(x,y) + \mathbb{E}(x)\mathbb{E}(y))}{7N-7}
-\dfrac{36(7N-6)}{7N-7}\mathbb{E}\left(x\right)\mathbb{E}\left(y\right)\\
& = \dfrac{6(7N-6)}{7N-7}(Cov(x,y) + \mathbb{E}(x)\mathbb{E}(y))
-\dfrac{36(7N-6)}{7N-7}\mathbb{E}\left(x\right)\mathbb{E}\left(y\right)
\end{aligned}
\end{equation}

\end{document}