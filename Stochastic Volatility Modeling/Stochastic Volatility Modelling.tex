\documentclass[a4]{article}
\usepackage{amsmath,amsthm,amssymb}
\usepackage{xcolor}
% Set up the images/graphics package
\usepackage{graphicx}
\setkeys{Gin}{width=\linewidth,totalheight=\textheight,keepaspectratio}
\graphicspath{{figures/}}
\usepackage{graphicx}
\title{Stochastic Volatility Modelling}
\author{Cliff}

% Make prettier tables.
\usepackage{booktabs}

% The units package provides nice, non-stacked fractions and better spacing
% for units.
\usepackage{units}

% The fancyvrb package lets us customize the formatting of verbatim
% environments.  We use a slightly smaller font.
\usepackage{fancyvrb}


% Small sections of multiple columns
\usepackage{multicol}



%%% Custom Commands
%---------------------------------------------------------------------------
% Concise referencing
\newcommand{\eqnref}[1]{\eqref{#1}}
\newcommand{\secref}[1]{Section \ref{#1}}
\newcommand{\figref}[1]{Figure \ref{#1}}
\newcommand{\lemref}[1]{Lemma \ref{#1}}
\newcommand{\corref}[1]{Corollary \ref{#1}}
\newcommand{\thmref}[1]{Theorem \ref{#1}}
% Real numbers
\newcommand{\Real}[1]{\mathbb{R}^{#1}}
% Complex numbers
\newcommand{\Complex}[1]{\mathbb{C}^{#1}}
% Integers
\newcommand{\Integer}[1]{\mathbb{Z}^{#1}}
% Rank operator
\DeclareMathOperator{\rank}{\textnormal{rank}}
% Vec operator
\newcommand{\vecop}{\textnormal{vec}}
% Norm
\newcommand{\norm}[1]{\left|\left|#1\right|\right|}
% Trace
\newcommand{\trace}{\textnormal{tr}}
% Range
\newcommand{\range}{\textnormal{range}}
% Partial derivative
\newcommand{\pd}[2]{\dfrac{\partial #1}{\partial #2}}
% Complete derivative
\newcommand{\dd}[2]{\dfrac{d #1}{d #2}}
% Complete derivative, second order
\newcommand{\dds}[2]{\dfrac{d^2 #1}{d {#2}^2}}
% Limit to N / N
\newcommand{\limover}[1]{\lim_{#1 \rightarrow \infty} \dfrac{1}{#1}}
% Display style sum
\newcommand{\dsum}{\displaystyle\sum}
% arg min and arg max
\newcommand{\argmax}[1]{\underset{#1}{\operatorname{arg~max}}}
\newcommand{\argmin}[1]{\underset{#1}{\operatorname{arg~min}}}
%---------------------------------------------------------------------------
\newtheorem{theorem}{Theorem}
\newtheorem{definition}{Definition}
\newtheorem{example}{example}
\newtheorem{corollary}{Corollary}
\begin{document}
\tableofcontents
\newpage
\maketitle
\section{Introduction}
\subsection{Characterizing a usable model-the Black-Sholes equation}
In trading floor it's really important to get a correct pricing, but the question is that how to evaluate a valid model. There are several standards,\par 
\begin{itemize}
	\item Consistent with the terminal payoff: $f(T,S(T)) = P(T,S(T)),\forall S$.\\
	\item Able to do risk-neutral hedging.\\
\end{itemize}
Let's consider a scenario of shorting options through delta-hedging, \par 
\begin{equation}
P\&L = \underbrace{-[P(t+\delta t, S + \delta S) - P(t,S)]}_\text{PnL from shorting option} + \underbrace{rP(t,S)\delta t}_\text{interest accrued from money account} - \underbrace{\Delta [(r-q)S\delta t -\delta S]}_\text{PnL from hedging}
\end{equation}\par 
and write as differential form and using $\Delta = \frac{\partial P}{\partial S}$\par 
\begin{equation}
\begin{aligned}
P\&L &=\frac{\partial P}{\partial t}dt - \frac{\partial P}{\partial S}dS - \dfrac{1}{2}S^{2}\frac{\partial^{2} P}{\partial S^{2}}(\frac{d S}{S})^{2} + rPdt -  \dfrac{\partial P}{\partial S} [(r-q)Sdt - dS]\\
&=  -\underbrace{[\frac{\partial P}{\partial t} + (r-q)S\dfrac{\partial P}{\partial S} -rP]dt}_\text{theta portion}- \underbrace{\dfrac{1}{2}S^{2}\frac{\partial^{2} P}{\partial S^{2}}(\frac{d S}{S})^{2} }_\text{gamma portion}
\end{aligned}
\end{equation}\par
\bigbreak 
\subsubsection{Single Asset Case}
\noindent Our daily PnL reads,\par 
\begin{equation}\label{PnL}
P\&L = -A(t,S)\delta t - B(t,S)(\frac{\delta S}{S})^{2}
\end{equation} \par 
In order to be risk-neutral, there should be some constraints on $A(t,S)$ and $B(t,S)$. First, $A$ and $B$ cannot be both positive or negative. Second, they should obey an equation: $\frac{\delta S}{S}=\pm \sqrt{-\frac{A(t,S)}{B(t,S)}}\sqrt{\delta t}$. If we assume underlying S satisfies a lognormal distribution, and we could have $\left\langle \left( \frac{\delta S}{S}\right)^{2}\right\rangle =\hat{\sigma}^{2}\delta t$, where $\hat{\sigma}$ is the average of historical volatility. \par 
\bigbreak 
Then we could get an approximation of $A(t,S)$ through $A(t,S)=-\hat{\sigma}^{2}B(t,S)$, and substitute it into equation (\ref{PnL}),\par 
\begin{equation}\label{PnL_vol}
P\&L = -\dfrac{S^{2}}{2}\dfrac{\partial^{2}P_{\hat{\sigma}}}{\partial S^{2}}\left(\dfrac{\delta S^{2}}{S^{2}}-\hat{\sigma}^{2}\delta t\right)
\end{equation}
\subsubsection{Multiple Assets Case}
\begin{equation}
P\&L = -A(t,S)\delta t - \frac{1}{2}\sum_{ij}\phi_{ij}(t,S)\dfrac{\delta S_{i}}{S_{i}}\dfrac{\delta S_{j}}{S_{j}}
\end{equation}\par 
where $\phi_{ij}(t,S) = S_{i}S_{j}\dfrac{d^{2}P}{dS_{i}dS_{j}}|_{t,S}$ and S denotes the vector of $S_{i}$. Using eigenvalue decomposition,\par 
\begin{equation}
\phi = T\mathcal{\varphi}T^{T}
\end{equation}\par 
furthermore, \par 
\begin{equation}
\begin{aligned}
\varphi = \{\dfrac{d^{2}P}{dS_{i}dS_{j}}\}_{i,j\in \{0,..,n\}}\\
and\\
T = (S_{i})_{i\in \{0,...,n\}}
\end{aligned}
\end{equation}
The gamma portion of our $P\&L$ can be written as\par 
\begin{equation}
\sum_{ij}\phi_{ij}\dfrac{\delta S_{i} }{S_{i}}\dfrac{dS_{j}}{S_{j}} = U^{T}\phi U = U^{T}T\varphi T^{T}U = (T^{T}U)^{T}\varphi (T^{T}U)
\end{equation}
where $U = \{\dfrac{\delta S_{i}}{S_{i}}\}$ and $\delta z_{k}=T_{k}^{T}U$. our PnL reads:\par 
\begin{equation}
P\&L = -A\delta t -\frac{1}{2}\sum_{k}\varphi_{k}\delta z_{k}^{2}
\end{equation}\par 
As in mono-asset case, the condition for our model to be usable is\par 
\begin{equation}	
A = -\frac{1}{2}\sum_{k}\varphi_{k}w_{k}
\end{equation}\par 
\noindent so our PnL reads:\par 
\begin{equation}
P\&L = -\frac{1}{2}\sum_{k}\varphi_{k}(\delta z_{k}^{2}-w_{k}\delta t)
\end{equation}\par 
Let us express A differently, so as to give our P\&L a more symmetric form.\par 
\begin{equation}
	A = -\frac{1}{2} \sum_{k}\varphi_{k}\omega_{k} = -\frac{1}{2}tr(\varphi \omega)=-\frac{1}{2}tr(T^{T}\phi T\omega )=-\frac{1}{2}tr(\phi T\omega T^{T})=-\frac{1}{2}tr(\phi C) = -\frac{1}{2}\sum_{ij}\phi_{ij}C_{ij}
\end{equation}\par 
where $C = T\omega T^{T}$ is a positive matrix by construction,\par 
\begin{equation}
A=-\frac{1}{2}\sum_{k}\
phi_{k}(\dfrac{\delta S_{i} }{S_{i}}\dfrac{dS_{j}}{S_{j}}-C_{ij}\delta t)
\end{equation}
Because C is a positive matrix, it can be interpreted as an (implied) covariance matrix.\par 
\subsubsection{Conclusion}
In the general case of multiple hedge instruments, the condition that our model is usable - no situation in which our carry P\&L is systematically positive or negative- is that there exists a positive break-even covariance matrix $ C(t,S),\forall S, \forall t$.\par 
In addition, only suitable model can be used in the trading purposes, which called market model.
\subsection{How (in)effective is delta hedging}
The attitude of this section is to verify the average and standard deviation of P\&L incurred over option's life.\par 
\noindent We start from the Black-Sholes assumption:\par 
\begin{itemize}
	\item The underlying follows a lognormal process with constant volatility $\sigma$ same as implied volatility in option pricing and risk management.\\
	\item The hedging interval is approaching to zero, $\delta \rightarrow 0$.
\end{itemize}
So, the sum of P\&Ls vanishes with probability one.\par 
However, in real-time life, the analysis should be more complex. Let's reduce the restrictions step by step.\par 
\bigbreak 
\noindent \textbf{Step 1}\par 
The dollar gamma weighted implied volatility equals future realized volatility over the option's life.\par 
\begin{equation}
\left\langle \int^{T}_{0}e^{-rt}S^{2}\dfrac{d^{2}P_{\hat{\sigma}}}{dS^{2}}\sigma_{t}^{2}dt\right \rangle = \left \langle \int^{T}_{0}e^{-rt}S^{2}\dfrac{d^{2}P_{\hat{\sigma}}}{dS^{2}}\hat{\sigma}^{2}dt\right \rangle \end{equation}\par 
with the above condition holds, our final P\&L is not biased on average and let us concentrated on dispersion.\par 
\bigbreak 
\noindent \textbf{Step 2}\par 
Assume that the option is delta-hedging daily at times $t_{i}: \delta t=1$ day. Recall (\ref{PnL_vol}):\par 
\begin{equation}
P\&L = -\sum_{i}e^{-rt_{i}}\dfrac{S_{i}^{2}}{2}\dfrac{d^{2}P_{\hat{\sigma}}}{dS^{2}}(t_{i},S_{i})(r_{i}^{2}-\hat{\sigma}^{2}\delta t)
\end{equation}\par 
where $r_{i}$ are daily returns, given by $r_{i}=\dfrac{S_{i+1}-S_{i}}{S_{i}}$. It is dependent on the option's payoff since it related with gamma.\par 
\bigbreak
\noindent \textbf{Step 3}\par 
\bigbreak 
In addition, we assume option's discounted dollar gamma is constant, and $e^{-rt_{i}}S^{2}\dfrac{d^{2}P_{\hat{\sigma}}}{dS^{2}}=\text{initial value: } S_{0}^{2}\dfrac{d^{2}P_{\hat{\sigma}}}{dS^{2}}(t_{0},S_{0})$. Let's write the daily return $r_{i}$ as:\par 
\begin{equation}
r_{i} =\sigma_{i}\sqrt{\delta t}z_{i},\quad \langle z_{i}\rangle =0,\quad \langle z^{2}_{i}\rangle =1
\end{equation}\par 
\bigbreak 
\noindent Let's assume $z_{i}$ is iid and independent with $\sigma_{i}$, our P\&L:\par 
\begin{equation}
P\&L = -\sum_{i}e^{-rt_{i}}\dfrac{S_{i}^{2}}{2}\dfrac{d^{2}P_{\hat{\sigma}}}{dS^{2}}(t_{i},S_{i})(\sigma_{i}^{2}z_{i}^{2}-\hat{\sigma}^{2}\delta t)
\end{equation}\par 
\bigbreak 
\noindent \textbf{Step 4}\par 
Let's assume $\sigma_{i}$ is time-homogeneous, so that, $\sigma_{i}$ is independent with i and has $\hat{\sigma}^{2}=\langle \sigma_{i}^{2}\rangle$. The variance of $\sum_{i}(\sigma_{i}^{2}z_{i}^{2}-\hat{\sigma}^{2})\delta t$ is given by:\par 
\begin{equation}
\begin{array}{l}
\left\langle\sum_{i j}\left(\sigma_{i}^{2} z_{i}^{2}-\widehat{\sigma}^{2}\right) \delta t\left(\sigma_{j}^{2} z_{j}^{2}-\widehat{\sigma}^{2}\right) \delta t\right\rangle \\
\\
=\sum_{i}\left(\left\langle\sigma_{i}^{4} z_{i}^{4}\right\rangle+\widehat{\sigma}^{4}-2 \widehat{\sigma}^{4}\right) \delta t^{2}+\sum_{i \neq j}\left\langle\sigma_{i}^{2} \sigma_{j}^{2} z_{i}^{2} z_{j}^{2}+\widehat{\sigma}^{4}-2 \widehat{\sigma}^{4}\right\rangle \delta t^{2} \\
\\
=\sum_{i}(2+\kappa) \widehat{\sigma}^{4} \delta t^{2}+\sum_{i \neq j}\left(\left\langle\sigma_{i}^{2} \sigma_{j}^{2}\right\rangle-\widehat{\sigma}^{4}\right) \delta t^{2} \\
\\
=\sum_{i}(2+\kappa) \widehat{\sigma}^{4} \delta t^{2}+\left(\left\langle\sigma^{4}\right\rangle-\widehat{\sigma}^{4}\right) \sum_{i \neq j} f_{i j} \delta t^{2} \\
\\
=\widehat{\sigma}^{4}\left(\sum_{i}(2+\kappa) \delta t^{2}+\Omega \sum_{i \neq j} f_{i j} \delta t^{2}\right)
\end{array}
\end{equation}\par 
where the kurtosis $\kappa$ and variance correlation function $f_{ij}$ is given by:\par 
\begin{equation}
\kappa=\frac{\left\langle\sigma_{i}^{4} z_{i}^{4}\right\rangle}{\widehat{\sigma}^{4}}-3, \quad f_{i j}=\frac{\left\langle\left(\sigma_{i}^{2}-\widehat{\sigma}^{2}\right)\left(\sigma_{j}^{2}-\widehat{\sigma}^{2}\right)\right\rangle}{\sqrt{\left\langle\sigma_{i}^{4}\right\rangle-\widehat{\sigma}^{4}} \sqrt{\left\langle\sigma_{j}^{4}\right\rangle-\widehat{\sigma}^{4}}}
\end{equation}
and dimensionless factor,
\begin{equation}
\Omega=\frac{\left\langle\sigma^{4}\right\rangle-\widehat{\sigma}^{4}}{\widehat{\sigma}^{4}}=\frac{\left\langle\sigma^{4}\right\rangle-\left\langle\sigma^{2}\right\rangle^{2}}{\left\langle\sigma^{2}\right\rangle^{2}}
\end{equation}
In the end, the standard deviation is given by\par 
\begin{equation}
\operatorname{StDev}(P \& L)=\left|\frac{S_{0}^{2}}{2} \frac{d^{2} P_{\widehat{\sigma}}}{d S^{2}}\left(t_{0}, S_{0}\right)\right| \sqrt{\widehat{\sigma}^{4}\left(\sum_{i}(2+\kappa) \delta t^{2}+\Omega \sum_{i \neq j} f_{i j} \delta t^{2}\right)}
\end{equation}
\bigbreak 
\noindent \textbf{Step 5}\par 
We know the vega and gamma has the following relationship according to the Black-Sholes formula\par 

\begin{equation}
\frac{d P_{\widehat{\sigma}}}{d \widehat{\sigma}}=S^{2} \frac{d^{2} P_{\widehat{\sigma}}}{d S^{2}} \widehat{\sigma} T
\end{equation}
Using Vega, the standard deviation of PnL is \par 
\begin{equation}\label{std_pnl}
\operatorname{StDev}(P \& L)=\left|\widehat{\sigma} \frac{d P_{\widehat{\sigma}}}{d \widehat{\sigma}}\right| \frac{1}{2 T} \sqrt{\sum_{i}(2+\kappa) \delta t^{2}+\Omega \sum_{i \neq j} f_{i j} \delta t^{2}}
\end{equation}
\subsection{The Black-Scholes case}
Let's first assume that S follows the lognormal Black-Scholes dynamics $\sigma_{i}$, is constant and equal to $\hat{\sigma}$. So, $\Omega = 0, \kappa = 0,f = 0$.
\begin{equation}\label{normstd}
StDev(P\&L) = \dfrac{1}{2N}\left|\hat{\sigma}\dfrac{dP_{\hat{\sigma}}}{d\hat{\sigma}}\right|
\end{equation}\par 
\noindent Note that $\dfrac{\hat{\sigma}}{\sqrt{2N}}$  is approximately the standard deviation of the historical volatility estimator. The historical volatility estimator is given by:\par 
\begin{equation}
\bar{\sigma}^{2} = \dfrac{1}{N\delta t}\sum_{i}\left(\dfrac{S_{i+1} - S_{i}}{S_{i}}\right)^{2}
\end{equation}\par 
In the Black-Scholes assumption, daily return is assumed as Gaussian distribution.\par 
\begin{equation}
\bar{\sigma}^{2}\approxeq \dfrac{\hat{\sigma}^{2}}{N}\sum_{i}z_{i}^{2}
\end{equation}\par 
And we know $Var(\hat{\sigma}^{2}) = \dfrac{2\hat{\sigma}^{2}}{N}$ since\par 
\begin{equation}
\begin{aligned}
Var(\hat{\sigma}^{2}) &= Var\left(\dfrac{\hat{\sigma}^{2}}{N}\sum_{i}z_{i}^{2}\right)\\
&=\dfrac{\hat{\sigma}^{4}}{N^{2}}\sum_{i}Var(z_{i}^{2})\\
&=\dfrac{\hat{\sigma}^{4}}{N}\left[E(z_{i}^{4})-E(z_{i}^{2})^{2}\right]
\end{aligned}
\end{equation}\par 
Then since the moment-generating-function of normal distribution is, $mgf(t)=\exp(\mu t + \dfrac{\sigma^{2}}{2})$ and let $\mu = 0$\par 
\begin{equation}
\begin{aligned}
mgf^{(1)}(t)&=\sigma^{2}t\exp(\dfrac{\sigma^{2}t^{2}}{2})\\
mgf^{(2)}(t)&=(\sigma^{2}+\sigma^{4}t^{2})\exp(\dfrac{\sigma^{2}t^{2}}{2})\\
mgf^{(3)}(t)&=(\sigma^{6}t^{3}+3\sigma^{4}t )\exp(\dfrac{\sigma^{2}t^{2}}{2})\\
mgf^{(4)}(t)&=(\sigma^{8}t^{4} + 6\sigma^{6}t^{2}+3\sigma^{4})\exp(\dfrac{\sigma^{2}t^{2}}{2})\\
mgf^{(4)}(0) = 3\sigma^{4}=3
\end{aligned}
\end{equation}
So, the variance of historical volatility estimator is\par 
$$
Var(\hat{\sigma}^{2})=\dfrac{\hat{\sigma}^{4}}{N}\left[3-1\right]=\dfrac{2\hat{\sigma}^{4}}{N}
$$\par 
\bigbreak 
\noindent The relative standard deviation $StDev(\bar{\sigma}^{2})/\langle\bar{\sigma}^{2}\rangle =\sqrt{\frac{2}{N}}$, \textcolor{red}{if it is not too large, the relative standard deviation of historical volatility estimator is half of this}, $\frac{1}{\sqrt{2N}}$.\par 
\bigbreak 
\noindent Given the above conclusion, the StDev(P\&L) is approximately vega multiplies the standard deviation of historical volatility estimator with the same hedging schedule.\par
\begin{example}
	Let's assume a one-year ATM option with $\hat{\sigma}=20\%, S=1, P = 7.97\%\quad N=250$. As a result, $\frac{1}{\sqrt(2N)}\approx0.045$. And there are 2 Golden Thumb: (1) $P_{\hat{\sigma}}\approx \frac{1}{\sqrt{2\pi}}S\hat{\sigma}\sqrt{T}$; (2)$\hat{\sigma}\dfrac{dP_{\hat{\sigma}}}{d\hat{\sigma}}\approx P$. And it implies the StDev(P\&L) = 0.045P, which is approximately $5\%$ premium. And the bid/ask spread is approximately 10 \%.
\end{example} 
\subsubsection{The real case}
\paragraph{Analysis}
In real life, the $\sigma_{i}$ is not constantly equal to $\hat{\sigma}$ and $f_{ij}$ will not vanish. We assume the dynamics is time-homogeneous. Then variance correlation function $f_{ij}$ is the function of the difference $|i-j|$.
\begin{equation}
\sum_{ij}f_{ij}\delta t^{2}\approx \int^{T}_{0}du\int^{T}_{0}dtf(t-u) = 2\int^{T}_{0}f(T-\tau)f(\tau)
\end{equation}\par 
We now have the from (\ref{std_pnl}):\par 
\begin{equation}
\label{real_std_pnl}
\begin{aligned}
StDev(P\&L)&\approx |\hat{\sigma}\dfrac{dP_{\hat{\sigma}}}{d\hat{\sigma}}|\frac{1}{2T}\sqrt{(2+\kappa)\frac{T^{2}}{N}+2\Omega\int^{T}_{0}(T-\tau)f(\tau)d\tau}\\
&=|\hat{\sigma}\dfrac{dP_{\hat{\sigma}}}{d\hat{\sigma}}|\sqrt{\dfrac{2+\kappa}{4N}+\dfrac{\Omega}{2T^{2}}\int^{T}_{0}(T-\tau)f(\tau)d\tau}
\end{aligned}
\end{equation}
\bigbreak 
\noindent Now we can investigate the two contributors to StDev(P\&L).\par 
\begin{itemize}
\item The first part: Let's assume the daily variances are constant, so $\Omega$ vanishes. It is similar in (\ref{normstd}), but the $\kappa$ is not zero.\\
\item The second part: the prefactor $\Omega$ quantifies the dispersion of daily variances while $f(\tau)$ quantifies how a fluctuation in daily variance $\sigma_{i}$ on day $t_{i}$ impacts daily variances $\sigma_{i+\tau}^{2}$ on subsequent days.
\begin{equation}
\Omega=\frac{\left\langle\sigma^{4}\right\rangle-\left\langle\sigma^{2}\right\rangle^{2}}{\left\langle\sigma^{2}\right\rangle^{2}}, \text{ and } f_{i j}=\frac{\left\langle\left(\sigma_{i}^{2}-\widehat{\sigma}^{2}\right)\left(\sigma_{j}^{2}-\widehat{\sigma}^{2}\right)\right\rangle}{\sqrt{\left\langle\sigma_{i}^{4}\right\rangle-\widehat{\sigma}^{4}} \sqrt{\left\langle\sigma_{j}^{4}\right\rangle-\widehat{\sigma}^{4}}}
\end{equation}\par 
Thus, if $\Omega$ vanishes slowly then the daily variance would be strongly correlated. If we say that $\sigma_{i}$ is higher than $\hat{\sigma}$, the daily variances $\sigma_{j}$ will be likely higher than $\hat{\sigma}$. It will cause the daily P\&L keeps the same sign which generating strong correlation among daily P\&Ls:\par 
\begin{equation}\label{PnL_vol}
P\&L = -\dfrac{S^{2}}{2}\dfrac{\partial^{2}P_{\hat{\sigma}}}{\partial S^{2}}\left(\dfrac{\delta S^{2}}{S^{2}}-\hat{\sigma}^{2}\delta t\right)
\end{equation}
and increasing the variance of our final P\&L. (see Figure \ref{fig:screenshot005} and Eq \ref{real_std_pnl})
\begin{figure}\label{fig1}
	\centering
	\includegraphics[width=0.3\linewidth]{screenshot005}
	\caption{Final PnL}
	\label{fig:screenshot005}
\end{figure}
\begin{example}
	Set $f(\tau)=1$, then we have the second piece of (\ref{real_std_pnl}) is $\frac{\Omega}{4}$. If $\Omega$ is small, the impact of this term is equivalent to the impact of a relative displacement of $\hat{\sigma}$ by $\hat{\sigma}\frac{\sqrt{\Omega}}{2}$, regardless of the number N of daily rehedges. (For this sentence consider the Taylor expansion on $\sqrt{\Delta x}$)
\end{example}
\end{itemize}
\paragraph{Estimating $f(\tau), \Omega, \kappa$}
\end{document}
