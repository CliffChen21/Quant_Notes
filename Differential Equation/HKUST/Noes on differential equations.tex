\documentclass[a4]{article}
\usepackage{amsmath,amsthm,amssymb}
\usepackage{xcolor}
% Set up the images/graphics package
\usepackage{graphicx}
\setkeys{Gin}{width=\linewidth,totalheight=\textheight,keepaspectratio}
\graphicspath{{figures/}}

\title{Notes on differential equation}
\author{Cliff}

% Make prettier tables.
\usepackage{booktabs}

% The units package provides nice, non-stacked fractions and better spacing
% for units.
\usepackage{units}

% The fancyvrb package lets us customize the formatting of verbatim
% environments.  We use a slightly smaller font.
\usepackage{fancyvrb}


% Small sections of multiple columns
\usepackage{multicol}



%%% Custom Commands
%---------------------------------------------------------------------------
% Concise referencing
\newcommand{\eqnref}[1]{\eqref{#1}}
\newcommand{\secref}[1]{Section \ref{#1}}
\newcommand{\figref}[1]{Figure \ref{#1}}
\newcommand{\lemref}[1]{Lemma \ref{#1}}
\newcommand{\corref}[1]{Corollary \ref{#1}}
\newcommand{\thmref}[1]{Theorem \ref{#1}}
% Real numbers
\newcommand{\Real}[1]{\mathbb{R}^{#1}}
% Complex numbers
\newcommand{\Complex}[1]{\mathbb{C}^{#1}}
% Integers
\newcommand{\Integer}[1]{\mathbb{Z}^{#1}}
% Rank operator
\DeclareMathOperator{\rank}{\textnormal{rank}}
% Vec operator
\newcommand{\vecop}{\textnormal{vec}}
% Norm
\newcommand{\norm}[1]{\left|\left|#1\right|\right|}
% Trace
\newcommand{\trace}{\textnormal{tr}}
% Range
\newcommand{\range}{\textnormal{range}}
% Partial derivative
\newcommand{\pd}[2]{\dfrac{\partial #1}{\partial #2}}
% Complete derivative
\newcommand{\dd}[2]{\dfrac{d #1}{d #2}}
% Complete derivative, second order
\newcommand{\dds}[2]{\dfrac{d^2 #1}{d {#2}^2}}
% Limit to N / N
\newcommand{\limover}[1]{\lim_{#1 \rightarrow \infty} \dfrac{1}{#1}}
% Display style sum
\newcommand{\dsum}{\displaystyle\sum}
% arg min and arg max
\newcommand{\argmax}[1]{\underset{#1}{\operatorname{arg~max}}}
\newcommand{\argmin}[1]{\underset{#1}{\operatorname{arg~min}}}
%---------------------------------------------------------------------------
\newtheorem{theorem}{Theorem}


\begin{document}

\maketitle

\begin{abstract}
\noindent Begin abstract
\end{abstract}



\section{Introduction}
\section{Partial Differential Equation}
\subsection{Fourier series}
\begin{equation}f(x)=\frac{a_{0}}{2}+\sum_{n=1}^{\infty}\left(a_{n} \cos \frac{n \pi x}{L}+b_{n} \sin \frac{n \pi x}{L}\right)\end{equation}
According to the orthogonality relations for sine and cosine. \par 
\begin{equation}\begin{array}{c}
\int_{-L}^{L} \cos \left(\frac{m \pi x}{L}\right) \cos \left(\frac{n \pi x}{L}\right) d x=L \delta_{n m} \\
\int_{-L}^{L} \sin \left(\frac{m \pi x}{L}\right) \sin \left(\frac{n \pi x}{L}\right) d x=L \delta_{n m} \\
\int_{-L}^{L} \cos \left(\frac{m \pi x}{L}\right) \sin \left(\frac{n \pi x}{L}\right) d x=0
\end{array}\end{equation}\par
\textcolor{blue}{***Note: detailed notes in Integral Notes$->$Step 1}\par
\bigbreak 
We multiply f(x) by $\sin \frac{n\pi x}{L}$ and take integral with respect to x.\par 

\begin{equation}\begin{aligned}
\int_{-L}^{L} f(x) \sin \frac{n \pi x}{L} d x &=\sum_{m=1}^{\infty} b_{m} \int_{-L}^{L} \sin \frac{n \pi x}{L} \sin \frac{m \pi x}{L} d x \\
&=\sum_{m=1}^{\infty} b_{m}\left(L \delta_{n m}\right) \\
&=L b_{n}
\end{aligned}\end{equation}
Then multiply f(x) by $\cos \frac{n\pi x}{L}$ and take integral with respect to x.\par 
\begin{equation}a_{n}=\frac{1}{L} \int_{-L}^{L} f(x) \cos \frac{n \pi x}{L} d x, \quad b_{n}=\frac{1}{L} \int_{-L}^{L} f(x) \sin \frac{n \pi x}{L} d x\end{equation}
\subsection{Fourier sine and cosine series}
We now know that the Fourier series of a periodic function with period $2 L$ is given by
$$
f(x)=\frac{a_{0}}{2}+\sum_{n=1}^{\infty}\left(a_{n} \cos \frac{n \pi x}{L}+b_{n} \sin \frac{n \pi x}{L}\right)
$$
with
$$
a_{n}=\frac{1}{L} \int_{-L}^{L} f(x) \cos \frac{n \pi x}{L} d x, \quad b_{n}=\frac{1}{L} \int_{-L}^{L} f(x) \sin \frac{n \pi x}{L} d x
$$
In this section, we provide a simplified version of fourier series if $f(x)$ is odd or even.\par 
\bigbreak 
If $f(x)$ is an even function,\par 
\begin{equation}a_{n}=\frac{2}{L} \int_{0}^{L} f(x) \cos \frac{n \pi x}{L} d x, \quad b_{n}=0\end{equation}
\begin{equation}f(x)=\frac{a_{0}}{2}+\sum_{n=1}^{\infty} a_{n} \cos \frac{n \pi x}{L}, \quad f(x) \text { even }\end{equation}\par 
\bigbreak
If $f(x)$ is an odd function,\par 
\begin{equation}a_{n}=0, \quad b_{n}=\frac{2}{L} \int_{0}^{L} f(x) \sin \frac{n \pi x}{L} d x\end{equation}
\begin{equation}f(x)=\sum_{n=1}^{\infty} b_{n} \sin \frac{n \pi x}{L}, \quad f(x) \text { odd }\end{equation}
\subsection{Example of Fourier series}
\begin{equation}f(x)=1-\frac{2 x}{\pi}, \quad 0<x<\pi\end{equation}
Fourier expression,\par
\begin{equation}f(x)=\frac{8}{\pi^{2}}\left(\cos x+\frac{\cos 3 x}{3^{2}}+\frac{\cos 5 x}{5^{2}}+\ldots\right)\end{equation}
\bigbreak
\noindent Since\par 
\begin{equation}f(x)=\frac{a_{0}}{2}+\sum_{n=1}^{\infty} a_{n} \cos n x, \text { with } a_{n}=\frac{2}{\pi} \int_{0}^{\pi} f(x) \cos n x d x\end{equation}
\begin{equation}a_{n}=\frac{2}{\pi} \int_{0}^{\pi}\left(1-\frac{2 x}{\pi}\right) \cos n x d x=\frac{4}{n^{2} \pi^{2}}(1-\cos n \pi)=\left\{\begin{array}{ll}
8 /\left(n^{2} \pi^{2}\right), & \text { if } n \text { odd } \\
0, & \text { if } n \text { even }
\end{array}\right.\end{equation}
\subsection{Diffusion equation (separation of variables)}
\begin{equation}\begin{array}{c}
u_{t}=D u_{x x}\\
u_{0,t} = 0\\
u_{L,t} = 0\\
u(x, 0)=f(x)
\end{array}\end{equation}
where,
\begin{equation}
D \text{  is a constant },\quad D>0.
\end{equation}\par
We assume,\par 
\begin{equation}u(x, t)=X(x) T(t)\end{equation}
and substitute the ansatz into the diffusion equation\par
\begin{equation}\begin{array}{c}X T^{\prime}=D X^{\prime \prime} T\\
\quad\\
\dfrac{X^{\prime \prime}}{X}=\dfrac{1}{D} \dfrac{T^{\prime}}{T}

\end{array}\end{equation} 
The left-hand side is independent to t and the right-hand side is independent to x. So, they need to be equal to a constant. Let's say $\lambda$.\par
\begin{equation}X^{\prime \prime}+\lambda X=0, \quad T^{\prime}+\lambda D T=0\end{equation}\par 
\noindent Then it is easy to solve these two ODEs with the boundary conditions\par 
\begin{equation}u(0, t)=X(0) T(t)=0, \quad u(L, t)=X(L) T(t)=0\end{equation}\par 
Solution. 1\par 

\begin{equation}
\begin{array}{c}
X^{\prime \prime}+\lambda X=0, \quad X(0)=X(L)=0\\
\quad \\
c^{2} + \lambda  = 0, \quad \lambda = \mu^{2}\\
c = \pm \mu i \\


X(x)=A \cos \mu x+B \sin \mu x\\
\quad \\
A = 0\quad (X(0) = 0),\quad B\sin \mu L = 0\quad (X(L) = 0)\\
\quad \\
\sin \mu L = 0\Rightarrow \mu_{n} = \dfrac{n\pi}{L}\\
\quad \\
X_{n}=\sin (n \pi x / L),\quad n = 1,2,3,...
\end{array}
\end{equation}\par 
Solution. 2\par 
\begin{equation}\begin{array}{c}
T^{\prime}+\left(n^{2} \pi^{2} D / L^{2}\right) T=0 \\
\quad \\
T_{n}=e^{-n^{2} \pi^{2} D t / L^{2}}
\end{array}\end{equation}\par 
In the end,\par 
\begin{equation}u_{n}(x, t)=\sin (n \pi x / L) e^{-n^{2} \pi^{2} D t / L^{2}}\end{equation}\par 
By superposition for homogeneous linear differential equations the states that the general solution with the spatial boundary conditions is given by\par 
\begin{equation}u(x, t)=\sum_{n=1}^{\infty} b_{n} u_{n}(x, t)=\sum_{n=1}^{\infty} b_{n} \sin (n \pi x / L) e^{-n^{2} \pi^{2} D t / L^{2}}\end{equation}
Then we can apply the last condition and if\par 

\begin{equation}f(x)=\sum_{n=1}^{\infty} b_{n} \sin (n \pi x / L)\end{equation}

\begin{equation}b_{n}=\frac{2}{L} \int_{0}^{L} f(x) \sin \frac{n \pi x}{L} d x\end{equation}
\subsection{Summary: Diffusion equation}
Given a diffusion equation,\par 
\begin{equation}\begin{array}{c}
u_{t}=D u_{x x}\\
u_{0,t} = 0\\
u_{L,t} = 0\\
u(x, 0)=f(x)
\end{array}\end{equation}\par 
where,
\begin{equation}
D \text{  is a constant },\quad D>0.
\end{equation}\par
\bigbreak
\bigbreak
\noindent \textsl{Solution.}\par 

\begin{equation}\begin{array}{c}
f(x)=\sum_{n=1}^{\infty} b_{n} \sin (n \pi x / L)\\
\quad\\
b_{n}=\frac{2}{L} \int_{0}^{L} f(x) \sin \frac{n \pi x}{L} d x
\end{array}\end{equation}
\end{document}
